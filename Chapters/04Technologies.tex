\documentclass[../Main.tex]{subfiles}
\begin{document}

\subsection{Neural Network}
\asia[inline]{describe NN technologies}
    \subsubsection{PyTorch}
    \subsubsection{TensorRT}
    \subsubsection{OpenCV}
    \subsubsection{CUDA}
    
\subsection{Server}
\bartek[inline]{describe server}
    \subsubsection{Python}
    \subsubsubsection{Description} 
    \subsubsubsection{Motivation}
    \subsubsubsection{Difficulties}

\subsection{Mobile Application}
    This part describes the style tranfer clinet, which is mobile application 
    for Android systems. We will go through all features of the application and
    explain why we've choosen Flutter framework. 
    \subsubsection{Flutter}
        Today there are several frameworks which allow us to create mobile 
        applications for both Android and iOS simultaneously, 
        eg. React Native, Ionic, Xamarin.
        In 2015 Flutter joined this family and changed drasticly situation
        on mobile application scene. At the moment when we write this article,
        Flutter is the most popular multiplatform framework, which allow us to 
        create not only mobile applications but also web from single codebase.
    
    \subsubsubsection{Description}
        Flutter is framework created by Google but it's also an open source project 
        so everone can propose new fetures and improvments. 
        When we write Flutter application we use Dart language.
        Dart is also developed by Google and has C-style syntax.
        It runs's on Dart VM included in the SDK. 
        Very interesting feautre of Dart is that it can transcompile to JavaScript.
        This feature is 
    
    There are Material Design 
        icons, colors, typical UI interface elements like top and bottom bars,
        buttons, list, animations and much more. 
        It all makes creating app with "professional" look very easy.
        
        
    \subsubsubsection{Motivation}
        From the begining of our project we wanted to be able to develop new ideas
        very quickly.
        Python and PyTorch are technologies which definitely helped us keep high
        productivity during this time. First choise was Kotlin - new mordern programing 
        language for Android apps. Main advantages of Kotlin are:
            \begin{itemize}
                \item officialy supported by Google
                \item high performance - native application
                \item some great language features like null safety, readabilty or async functions
                \item we had some experience with this langauge
            \end{itemize}
            
        The main drawback of creating mobile apps with Kotlin is that creating 
        some simple things can often take much time and effort. 
        This was a main cause of rejecting this approach.
        We've looked for some other tools to create our clinet application and 
        we've found Flutter - relatively new tool framework from Google which 
        is still developing very quickly.
        The main advantage of Flutter is that you can develop your apps very 
        quckly beceause there are tons of ready, efficient and highly costumizable 
        widgets. There is also large community which creates new great widgets 
        add publishes them on official website so everyone can dowload them and use in project.
        You can also modify or build new widgets on top of it and then publish.
        If some widget would become very popular and usefull it will may be 
        include to official Flutter SDK. 
        Flutter also makes easy to create apps in beatiful Mateterial Design 
        style or it's iOS equivalent - Cupertino. 
        The main feature of Flutter, and always mentioned by Google is of course 
        creating apss for Android and iOS at the same time. 
        Our main goal was to create app for Android so we treated this feature like 
        an opened door. 
        Flutter is also declarative, react style framework which we prefere while
        creating user interface's. We've had previous experience with React on web
        so Flutter seemed easy to learn.  
        
    \subsubsubsection{Difficulties}



\biblio % Needed for referencing to working when compiling individual subfiles - Do not remove
\end{document}

