\documentclass[../Main.tex]{subfiles}
\begin{document}

\subsection{What is style transfer}

\subsection{Machine Learning} 
% or DEEP LEARNING
\bartek[inline]{write intros based on artciles listed below}
    \subsubsection{What is Machine Learning?}
    see chapter 1.1 in \\
    \url{https://www.bc.edu/content/dam/bc1/schools/mcas/cs/pdf/honors-thesis/Aleshin-Guendel_Serge_Thesis.pdf} \\
    see chapter 2.1.1 in \\
    \url{https://www.ke.tu-darmstadt.de/lehre/arbeiten/master/2016/Muenker_Christoph.pdf} \\
    check \\
    \url{https://www.ics.ei.tum.de/fileadmin/w00bcw/www/ics/pics/research/data_sets/MasterarbeitICS-FINALVERSION-Niklas.pdf}\\


\subsection{Neural Networks}
    \subsubsection{Overview}
    \subsubsection{Layers}
        \textbf{Fully Connected Layers}
        \textbf{Convolutional Layers}
    \subsubsection{Loss} %for style tranfer
    \subsubsection{Activation}
        \textbf{ReLU}
    \subsubsection{Backpropagation}

\subsection{Neural network compression}
    Latest neural networks usually have between 2 million and 50 million parameters.
    Older architectures are even heavier - full VGG16 model has as much as
    138 million parameters. In consequence models 
    are often too big for deployment on memory bounded mobile and embedded devices.
    Number of parameters strongly correlates with inference time, which in turn
    prevents real-time inference not only on embedded and mobile
    devices but even on middle-class GPUs. Overcoming these limitations 
    is very active area of research.\\
    \textbf{Specialized architectures} such as MobileNet [\cite{mobilenetv1},
    \cite{mobilenetv2}, \cite{mobilenetv3}] and ShuffleNet [\cite{shufflenetv1},
    \cite{shufflenetv2}], replace full convolutions with bottlenecks of lighweight
    pointwise, depthwise and group convolutions in order to reduce number of 
    arithmetic operations and parameters. MobileNetV3 \cite{mobilenetv3} uses
    Neural Architecture Search to optimize the network for mobile phone CPU
    inference, while discussion in \cite{shufflenetv2} shows what aspects should
    be considered, when designing mobile architecture manually. 
    \textbf{Network pruning} aims to reduce already trained model's size by pruning away
    weights with the least impact on network's quality. [\cite{zhu2017prune}] show
    that for some models even up to \[87.5\%\] weights can be removed with only 
    marginally reduced quality. 
    Beacuse networks are initialized randomly, the least
    important parameters are usually spread across whole network. Naive pruning 
    then results in sparse networks. Their storage is significantlly reduced, however
    beacuse available linear algebra libraries are optimized for dense structures,
    their inference time doesn't scale as well. To overcome this, more structured
    methods of pruning were developed, among them filter pruning [\cite{li2016pruning}]
    and channel pruning [\cite{he2017channel}]. Weight's or weights set's importance
    can be measured by various heuristics. 
    [\cite{li2016pruning}] prune away th filters with the smallest \[L1\] norm.
    [\cite{polyak2015}] on the other hand choose to prune away the filters with
    the smallest activation statistics.


\subsection{Image and Video Processing}
    \subsubsection{Basics}
    \TODO {add subsections of Image and Video Processing}
    \subsubsection{Examples}


\biblio % Needed for referencing to working when compiling individual subfiles - Do not remove
\end{document}
